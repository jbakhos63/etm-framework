\documentclass[12pt]{article}
\usepackage{amsmath,amssymb}
\usepackage{tabularx}
\usepackage{graphicx}
\usepackage{hyperref}
\usepackage{geometry}
\geometry{margin=1in}

\title{Euclidean Timing Mechanics: Logic Notebook}
\author{Joseph Bakhos}
\date{\today}

\begin{document}

\maketitle

\tableofcontents
\newpage

\section{Axioms}

\subsection*{A1 [L]}
Nodes are the fundamental entities of ETM. Each node possesses an internal tick counter and phase state.

\subsection*{A2 [L]}
Time progresses through discrete ticks. No continuous time exists in the ETM ontology.

\subsection*{A3 [L]}
Return eligibility, identity persistence, and reinforcement behavior may reference ancestry and echo memory fields not spatially local to the node under evaluation.

\subsection*{Axiom A4 [L] — Phase-Coherent Identity Reformation}
A modular identity may reform in a node only if the following three logical conditions are satisfied:
\begin{enumerate}
  \item \textbf{Ancestry Match:} The identity's ancestry tag $A_i$ satisfies $match(A_i, A_r) = \text{true}$ with the recruiter's ancestry tag $A_r$.
  \item \textbf{Phase Tolerance:} The identity's phase $\theta_i$ satisfies $|\theta_i - \theta_r| \le \delta\theta$ (modulo 1), where $\theta_r$ is the recruiter phase and $\delta\theta$ is the global return tolerance constant.
  \item \textbf{Echo Threshold:} The recruiter’s echo support satisfies $\rho_r \ge \rho_{\text{min}}$, the minimum required for reformation.
\end{enumerate}
This axiom governs all return logic within ETM and forms the basis for modular rhythm re-entry behavior.
\textit{Confirmed in Trials 001 (success), 002 (failure), and 003 (threshold sweep).}

\subsection*{Axiom A4b [L] — Refined Echo Scope for Return Evaluation}
In addition to phase and ancestry alignment, the echo reinforcement condition in A4 is refined as follows:
\begin{itemize}
  \item Let $\rho_{\text{local}}$ denote the echo at the rotor’s own node.
  \item Let $\rho_{\text{neigh}}$ denote the sum of echoes in the 1-hop neighborhood.
\end{itemize}

Return is permitted if either:
\begin{enumerate}
  \item $\rho_{\text{local}} \ge \rho_{\text{min}}$, or
  \item $\rho_{\text{neigh}} \ge \rho_{\text{min}}$, or
  \item A hybrid weighted average satisfies:
  \[
  0.6 \cdot \rho_{\text{local}} + 0.4 \cdot \rho_{\text{neigh}} \ge \rho_{\text{min}}
  \]
\end{enumerate}

This refinement prevents failure in cases where echo reinforcement is distributed rather than concentrated.

\textit{Supported by Trials 004 and 005.}

\paragraph{Phase Constraint:} Return is additionally gated by tick-phase matching. The modular identity’s phase must match the recruiter’s phase within:
\[
|\theta_{\text{return}} - \theta_{\text{recruiter}}| \leq 0.11 \ (\text{mod} \ 1.0)
\]

\subsection*{A5 [L] – Symbolic Identity Conflict Gate}
No two modular identities with identical ancestry and identical return phase may simultaneously reform into the same modular rhythm structure unless a differentiating condition exists.\\
\textbf{Source:} Trials 014–016

\subsection*{A6 [L] – Modular Phase Distinction Principle}
Two modules with identical ancestry can remain distinct if their recruiter fields enforce a stable tick rhythm offset. This distinction underlies orbital state separation (e.g., ground and excited) in ETM.

\textit{Confirmed in Trial 019.}

\subsection*{A7 [L] – Recruiter Rhythm Dominance Principle}
Return into a recruiter-defined identity module is governed by phase alignment. Rotor ancestry does not override recruiter rhythm match.
\textit{Confirmed in Trials 023–024.}

\section{Derived Rules}

\subsection*{R1 [L]}
A modular identity may only re-form at a node if ancestry match and recruiter phase coherence are satisfied within tolerance.

\subsection*{R2 [L]}
Phase state updates follow a modular rhythm:
\begin{equation}
\phi_{t+1} = (\phi_t + \Delta\phi) \bmod 1.0
\label{eq:phase_update}
\end{equation}
where $\Delta\phi$ is identity-specific.

\subsection*{Rule R3 [L] — Return Window Disqualification}
\textit{Derived from A4.}  
Return is prohibited at a given tick if any of the three required conditions in Axiom A4 are not satisfied. That is:
\[
\neg(\text{Ancestry Match}) \lor \neg(\text{Phase Alignment}) \lor \rho_r < \rho_{\text{min}} \Rightarrow \text{Return Disallowed}
\]
This rule enforces strict reformation logic and defines one of the key gating mechanisms for identity evolution in ETM.

\subsection*{Rule R4 [L] — Static Echo Deficit Rule}
If a rotor remains stationary and no external identity propagates through its location, then:
\[
\lim_{t \to \infty} \rho_{\text{local}}(t) < \rho_{\text{min}}
\]
This explains the failure of static return in environments without active echo cycling, as confirmed in Trial 005.
\textit{Demonstrated by echo plateauing in Trial 005 static rotor phase.}

\subsection*{Rule R5 [L] — Neighbor Echo Inheritance}
To model sustained return in distributed recruiter fields, we define an echo inheritance rule:
\[
\rho_{\text{local}}(t+1) = \rho_{\text{local}}(t) + \alpha \cdot \rho_{\text{neigh}}(t)
\]
where \( \alpha \) is a configurable inheritance factor (e.g., \( \alpha = 0.10 \)).

This rule enables local echo to build over time through steady reinforcement from the neighborhood, even in the absence of direct rotor reinforcement.

\textit{Validated in Trial 009.}

\subsection*{Rule R6 [L] – Identity Return Lockout (Pauli Conflict Rule)}

If two modular identities:
- Share the same ancestry tag,
- Attempt return into the same node at the same tick,
- Have aligned tick-phase,

then both identities are denied return. A lockout is enforced for that node and tick.  
\textit{Confirmed in Trial 013.}

\subsection*{R7 [L] – Phase Offset Escape}
Modular identities with identical ancestry may coexist within the same recruiter basin if their timing phases are sufficiently offset at reformation.\\
\textbf{Derived from:} A5, A4b \\
\textbf{Supported by:} Trial 015

\subsection*{R8 [L] – Phase-Gated Return Logic}
An identity rotor may reform into a recruiter-defined module only if its current phase matches the recruiter’s tick rhythm phase within an acceptable tolerance window. This return condition supports stable transitions between orbital identity states.

\textit{Derived from A4b and A6. Confirmed in Trial 019.}

\begin{itemize}

\item \textbf{R9 (Return Window Match)} A rotor may return into a recruiter-defined modular rhythm if and only if the difference between its tick-phase and the recruiter’s central phase (modulo 1) is within the maximum phase tolerance:
\begin{equation}
\delta\theta \le 0.11
\label{eq:return_window_tolerance}
\end{equation}
This rule supersedes ancestry or rotor type when recruiter coherence is dominant, but only within fully matching ancestry domains.

\item \textbf{R10 (Ancestry-Gated Return)} If symbolic ancestry matching is enabled, a rotor may return into a recruiter-defined rhythm only if the recruiter's ancestry symbol matches the rotor's ancestry.
\begin{equation}
\texttt{ancestry}(R) = \texttt{ancestry}(Q)
\label{eq:ancestry_gate}
\end{equation}
This condition overrides phase match in contexts where modular identity coherence is enforced.

\section*{Rule R11 [L] — Symbolic Mutation Smoothing}

If ancestry mismatch exceeds the normal return threshold, identity return may still be allowed if a symbolic smoothing rule reduces the effective mismatch below the gating ceiling.

Symbolic smoothing may include:
\begin{itemize}
    \item Wildcard alignment (e.g., \texttt{X} $\rightarrow$ \texttt{*})
    \item Mutational forgiveness layers
    \item Tagged equivalence classes
\end{itemize}

This rule enables return for identities with structurally repairable ancestry under conditions where:
\begin{itemize}
    \item $\rho_{\text{hybrid}} \ge \rho_{\min}$
    \item $|\theta_{\text{return}} - \theta_{\text{recruiter}}| \le 0.11 \ (\text{mod} \ 1.0)$
\end{itemize}

\textbf{Derived from:} A4, R9, R10 \\
\textbf{Confirmed in:} Trial 049


\end{itemize}

\section{Definitions}

\begin{itemize}
    \item \textbf{Tick}: A discrete unit of local time.
    \item \textbf{Phase}: A modular value in $[0.0, 1.0)$ used for resonance and rhythm alignment.
    \item \textbf{Recruiter}: A structure formed by surrounding nodes that reinforces identity or enables return.
    \item \textbf{Echo Field}: A transient ancestry-bearing rhythm memory that may influence node behavior.
\end{itemize}

\section{Open Questions}

\begin{itemize}
    \item Can A3 reproduce all empirically observed Bell-type correlations?
    \item What is the minimal set of axioms required to derive Pauli-style exclusion behavior?
\end{itemize}

\end{document}
